\subsubsection{SSH}

\subsubsubsection{Create SSH keys}

Create SSH keys by entering a passphrase connected to the SSH keys, after running:

\code{ssh-keygen -f ~/.ssh/id_ecdsa -t ecdsa -b 521}

\subsubsubsection{Remember passphrase}

By using \code{ssh-agent}, you do not have to enter the SSH-key passphrase everytime.

Append the following to the \code{~/.profile}

\begin{minted}[]{bash}
SSH_ENV="$HOME/.ssh/agent-environment"

function start_agent {
    echo "Initialising new SSH agent..."
    /usr/bin/ssh-agent | sed 's/^echo/#echo/' > "${SSH_ENV}"
    echo succeeded
    chmod 600 "${SSH_ENV}"
    . "${SSH_ENV}" > /dev/null
    /usr/bin/ssh-add;
}

# Source SSH settings, if applicable

if [ -f "${SSH_ENV}" ]; then
    . "${SSH_ENV}" > /dev/null
    #ps ${SSH_AGENT_PID} doesn't work under cywgin
    ps -ef | grep ${SSH_AGENT_PID} | grep ssh-agent$ > /dev/null || {
        start_agent;
    }
else
    start_agent;
fi
\end{minted}

then \code{source ~/.profile} and the \code{ssh-agent} will ask for the passphrase and then you do not have to enter it again. The script will look for the agent automatically.

You can manually add a SSH-key with:

\code{ssh-add ~/.ssh/id_ecdsa}

\subsubsubsection{Github}

Copy the public key, that is all of the content in \code{id_ecdsa.pub}. If in WSL Ubuntu, copy the content with:

\code{clip.exe < ~/.ssh/id_ecdsa.pub}

Go to \href{www.github.com}{Github}, then: 

\texttt{Settings \ra\ SSH and GPG keys \ra\ Add copied public key}

Now you can clone a repo:

\code{git clone git@github.com:<username>/<reponame>.git}

Say yes to the fingerprint.
