\subsubsection{Docker for WSL2}

\textbf{In Windows:}

\link{https://hub.docker.com/editions/community/docker-ce-desktop-windows/}{Download Docker Desktop}

Install and make sure that the following checkboxes are ticked:
\begin{itemize}
    \item Enable Hyper-V Windows Features (Probably not necessary because of using WSL2 and not WSL1, but won't hurt)
    \item Install required Windows components for WSL 2
\end{itemize}

Run the Docker Desktop program in Windows. If the engine never starts, try to reboot your computer.

In Docker Desktop go into settings through the gear icon in the top right and then:
\begin{itemize}
    \item \textbf{General} make sure that
    \begin{itemize}
        \item The \textbf{Use the WSL 2 based engine} checkbox is checked.
    \end{itemize}
    \item \textbf{Resources} \ra \textbf{WSL INTEGRATION} and make sure that
    \begin{itemize}
        \item The \textbf{Enable integration with my default WSL distro} checkbox is checked.
        \item Your installed distribution is enabled.
    \end{itemize}
\end{itemize}

Open up Ubuntu and check if docker is available:

\code{docker --version}

\ifdefined\wslRelevantDockerTest
\subimport{test/}{body.tex}
\fi