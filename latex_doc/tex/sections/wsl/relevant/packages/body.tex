\subsubsection{Packages}

\subsubsubsection{trash-cli - Remove files with possibility to restore}

Usually when using the \code{rm} command, it is very hard to restore what has been removed. By using this package, a trashbin is being used before removing permanently.

\begin{enumerate}
    \item Install the \code{trash-cli} package with \code{sudo apt install trash-cli}
    \item Then set the alias \code{alias rm=trash} in your \code{~/.bashrc} file.
\end{enumerate}


The \code{trash-cli} package comes with some commands:
\begin{itemize}
    \item \code{trash} - Which is like \code{rm} but it will be put in a trashbin instead
    \item \code{trash-list} - Lists everything in the trashbin
    \item \code{restore-trash} - Lists and numbers everything in the trashbin. Asks for an index from the list to restore.
    \item \code{trash-empty} - Permanently remove everything in the trashbin.
\end{itemize}


\subsubsubsection{apparix - Terminal bookmark directories}

Run \code{apparix} in order for it to set up its folders.

Then write \code{apparix --shell-examples} and copy everything except the aliases at the bottom. Paste this in \mintinline{text}{|\textapprox|/.bashrc}

If just copying from the terminal and pasting into \code{~/.bashrc} doesn't work, create a file and let the output be written into that instead in order to copy from it.

\code{touch text.txt}

\code{apparix --shell-examples > text.txt}

Then paste it into \texttt{~/.bashrc}.

Restart console.

Bookmark current directory with \code{bm bookmarkname} and go to the same location with \code{to bookmarkname}