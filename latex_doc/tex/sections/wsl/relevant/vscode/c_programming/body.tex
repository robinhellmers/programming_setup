\subsubsubsection{C programming}

\begin{enumerate}
    \item Install the necessary tools by running
    \begin{itemize}
        \item \code{sudo apt install build-essential} for \texttt{gcc}
        \item \code{sudo apt install gdb} for \texttt{gdb}
    \end{itemize}

    \item Create a new folder in the WSL where you create a C file \texttt{helloworld.c}. New folder is necessary for Visual Studio Code to realise that there is a C compiler to setup later on as it uses the open file to do the configurations.
    
    \item Open up Visual Studio Code
    
    \item Install the \texttt{C/C++} extenstion from Microsoft.
    
    \item Open up the folder you created the main C file in. \texttt{File\ra Open Folder...}
    
    \item Open up the \texttt{helloworld.c} file in the file explorer.
    
    
\end{enumerate}


bla


\begin{enumerate}
    \setcounter{enumi}{783}

    \item asd
\end{enumerate}



\subsubsubsubsection{Single file compilation}

\subsubsubsubsection{Multi-file compilation}

Here is some info about setting up Visual Studio Code to build and debug projects including multiple files:

\link{https://dev.to/talhabalaj/setup-visual-studio-code-for-multi-file-c-projects-1jpi}{https://dev.to/talhabalaj/setup-visual-studio-code-for-multi-file-c-projects-1jpi}

\subsubsubsubsection{Global multiple word search}

Some times you might want to find a specific file or line of code with multiple words in it, without
having to be in a direct sequence. Use this extension which automates the process of using \texttt{regex}.

\textbf{Search} by Alexander:

\link{https://marketplace.visualstudio.com/items?itemName=usernamehw.search}{https://marketplace.visualstudio.com/items?itemName=usernamehw.search}