\subsubsection{Setup of the distribution}

When installing, you will be prompted for a username.

\subsubsubsection{If the username can't be chosen}

If the username can't be chosen because of using e.g. a dot (.), press \code{Ctrl + C} and the window should close.

Run the \code{.exe} file again and you should now have skipped the first user initialization and be logged in as \code{root}.

\code{adduser -aG --force-badname <username>}

Set a password and you can skip entering all of your details by just pressing enter till it's done.

Now we are going to give the new user \code{sudo} privileges by adding it to the \code{sudo} group:

\code{usermod -aG sudo <username>}

Verify that the new user is in the \code{sudo} group with: \code{groups <username>}

Which should show: \code{<username> : <username> sudo}

Log into the new user:

\code{su - <username>}

\ifdefined\wslDistroSetupDefaultUser
\subimport{default_user/}{body.tex}
\fi

%%%%%%%%%%%%%%%%%%%%%%%%%%%%%
%%%%% Update the system %%%%%
%%%%%%%%%%%%%%%%%%%%%%%%%%%%%

\subsubsection{Update the system}

Update the list of packages, but doesn't install:

\code{sudo apt update}

Install new version of packages and say yes to every question:

\code{sudo apt upgrade -y}

%%%%%%%%%%%%%%%%%%%%%%%%%%%%%%%%
%%%%% Set WSL to version 2 %%%%%
%%%%%%%%%%%%%%%%%%%%%%%%%%%%%%%%

\subsubsection{Set WSL to version 2}

We are now going to check which WSL version which is used with the newly installed Ubuntu distribution.

Close any Ubuntu terminal open.

Open up the Command Prompt \code{cmd} and run:

\code{wsl -l -v}

Which should show all WSL distributions and which version they are running.

If your Ubuntu WSL version is 1, close any Ubuntu window and then run:

\code{wsl --set-version <distro name> 2}

After it is done, recheck that the version now is 2:

\code{wsl -l -v}