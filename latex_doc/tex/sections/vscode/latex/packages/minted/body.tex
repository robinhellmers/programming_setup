\subsubsubsection{Minted - Code viewing}
In order to use the \texttt{minted} package for displaying code one must install \texttt{Pygments} for \texttt{Python}.

If using \texttt{Ubuntu 18.04}:

\code{sudo apt install python-pygments}

If using \texttt{Ubuntu 20.04}:

\code{sudo apt install python3-pygments}

Also, one must use the \code{-shell-escape} flag with the \code{latexmk} (standard compilation) command in order to compile with \texttt{minted}.

In Visual Studio Code, press \texttt{Ctrl + Shift + P} in order to search for commands. Search and execute \code{Preferences: Open settings (JSON)} in order to open up the \texttt{settings.json} file with the compilation recipe. Add \code{"-shell-escape"} to the arguments.

If the file is empty, use the \texttt{settings.json} from \href{https://github.com/robinhellmers/computer_setup/blob/master/latex-workshop/settings.json}{Github}.

\begin{minted}[tabsize=3,obeytabs,linenos,bgcolor=codegray]{json}
{
    "name": "latexmk",
    "command": "latexmk",
    "args": [
        "-shell-escape",
        "-synctex=1",
        "-interaction=nonstopmode",
        "-file-line-error",
        "-pdf",
        "-outdir=%OUTDIR%",
        "%DOC%"
    ],
    "env": {}
}
\end{minted}

If there there is an error message similar to

\mintinline{text}{Undefined control sequence. \PYG #1#2->\FV@PYG}

\tttbf{WARNING:} Slows down compilation significantly!

Add the argument \code{cache=false} when loading the \code{minted} package.

\code{\usepackage[cache=false]{minted}}
