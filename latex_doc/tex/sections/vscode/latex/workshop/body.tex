\subsubsection{Latex Workshop}

Open up the settings \texttt{JSON} file:

\code{Ctrl + Shift + P} and enter

\code{>Preferences: Open Settings (JSON)}

Some times, the recipes does not appear in the \texttt{JSON} file. Append the code below if it doesn't exist. 

\begin{minted}{latex}
"latex-workshop.view.pdf.viewer": "browser",
"latex-workshop.latex.autoBuild.run": "onSave",
"latex-workshop.latex.autoClean.run": "onBuilt",
"latex-workshop.latex.recipe.default": "first",
"latex-workshop.chktex.run": "onType",
"latex-workshop.chktex.delay": 2000,
"latex-workshop.latex.recipes": [
    {
        "name": "latexmk",
        "tools": [
        "latexmk"
        ]
    },
    {
        "name": "pdflatex - bibtex - pdflatex x2",
        "tools": [
        "pdflatex",
        "bibtex",
        "pdflatex",
        "pdflatex"
        ]
    }
],
"latex-workshop.latex.tools": [
    {
        "name": "latexmk",
        "command": "latexmk",
        "args": [
        "-synctex=1",
        "-interaction=nonstopmode",
        "-file-line-error",
        "-pdf",
        "-shell-escape",
        "-outdir=%OUTDIR%",
        "%DOC%"
        ],
        "env": {}
    },
    {
        "name": "pdflatex",
        "command": "pdflatex",
        "args": [
        "-synctex=1",
        "-interaction=nonstopmode",
        "-file-line-error",
        "%DOC%"
        ],
        "env": {}
    },
    {
        "name": "bibtex",
        "command": "bibtex",
        "args": [
        "%DOCFILE%"
        ],
        "env": {}
    }
]
\end{minted}

\subsubsubsection{Default PDF viewer}

The setting

\code{"latex-workshop.view.pdf.viewer": "browser"}

will open up a browser tab when pressing the \texttt{View LaTeX PDF file} button. Using the browser works good with \code{Synctex} for jumping between PDF and code seamlessly.

\subsubsubsection{Synctex - Jump between PDF and code}

By having the flag \code{"-synctex=1"} in the recipe, one can enable jumping between the PDF and code locations seamlessly. Works good when using \code{"browser"} as default PDF viewer.

\begin{itemize}
    \item \code{Ctrl} clicking in the PDF.
    \item Set text marker in code and press \code{Ctrl + Alt + J}
\end{itemize}

\subsubsubsection{Default recipe}

The setting

\code{"latex-workshop.latex.recipe.default": "first"}

will run the first/top recipe given in \code{"latex-workshop.latex.recipes":}. It can be changed to \code{lastUsed} but might confuse some times.

Make sure that \code{latexmk} is the first item.


\subsubsubsection{Clean aux files}

The setting

\code{"latex-workshop.latex.autoClean.run": "onBuilt"}

will remove the \code{aux} files generated from the compilation after that the compilation is done.


\subsubsubsection{Chktex - Linting}

The settings

\code{"latex-workshop.chktex.run": "onType"}\\
\code{"latex-workshop.chktex.delay": 2000}

will enable linting using \code{Chtex}, checking two seconds after stopped writing. Will show problems in the \texttt{Problems} tab.

